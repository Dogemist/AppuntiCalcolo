
\documentclass[12pt]{article}
\usepackage{tikz}
\usepackage[utf8]{inputenc}
\usepackage{amsmath}

\newcommand\aug{\fboxsep=-\fboxrule\!\!\!\fbox{\strut}\!\!\!}

\title{Appunti sugli esercizi di Calcolo Numerico}


\begin{document}
\begin{titlepage}
\maketitle
\end{titlepage}

\section*{Premessa}

\label{Premessa}
Lo scopo di questo documento è quello di racchiudere le informazioni principali su quello che spesso (se non sempre) viene richiesto dalla professoressa Zaglia.
\\
Sono stati riportati principalmente gli argomenti con le loro formule e il procedimento per risolverli e applicarli. \\
In alcuni punti viene svolto un esempio in modo da mettere in pratica la teoria mentre su altri non viene riportato dato che basta solo sostituire alla formula i dati trovati (Ex Cholesky).  
\\ \\  Questi appunti \textbf{non} rispecchiano completamente il corso della prof, tanto meno quando (se lo farà) ci sarà il cambio di cattedra. Prendeteli quindi come spunto per capire e per applicare le cose. 
\\ Infine non sono una macchina (ci sto ancora lavorando), quindi il documento potrebbe essere soggetto ad errori :$>$

\begin{figure}
\centering
  \includegraphics[width=0.5\linewidth]{img/doggo.jpg}
  \label{doggo}
\end{figure}

\newpage

\tableofcontents

\newpage

\section{Equazioni Lineari}
\label{Equazioni Lineari}

Metodi per ottenere l'approssimazione di equazioni non lineari. In particolare consiste nel determinare
\begin{itemize}
	\item Gli zeri (o radici) della funzione $f(x) = 0$
	\item I valori (o Punti fissi) della funzione $x = g(x)$
\end{itemize}

\subsection{Interpretazione grafica}
\label{Interpretazione grafica}
Utile per individuare se e (in caso affermativo) dove ci sono le soluzioni.
Diversi dei metodi che verranno utilizzati richiederanno la conoscenza di un intervallo $[a,b]$ che contenga una soluzione \textbf{reale} ed \textbf{unica}.
\\ \\
L'interpretazione grafica ci aiuta molto in questo caso, per rappresentare solitamente "basta" disegnare le funzioni in questo modo:
\\ \\
$\left\{
  \begin{array}{lr}
    y=f(x) \\
    y=0
  \end{array}
\right.$
\\
Se la funzione è f(x), le soluzioni sono uguali alle intersezioni di f(x) con l'asse delle x
\\ \\ \\
$\left\{
  \begin{array}{lr}
    y=g(x) \\
    y=x
  \end{array}
\right.$
\\
Se la funzione è x=g(x), le soluzioni sono uguali alle intersezioni tra le due funzioni

\newpage

\subsection{Bisezione}
\label{Bisezione}
Consiste, partendo da un intervallo, nel dividere di volta in volta a metà l'intervallo di partenza fino a quando non si raggiunge un intervallo piccolissimo che è sempre più vicino alla soluzione x.
\\ \\
\textbf{Esempio Time}: data la funzione $$f(x)=3x - cos(x)$$ 
inizio a determinare graficamente le soluzioni attraverso il seguente sistema:
\\ \\
$\left\{
  \begin{array}{lr}
    y=cos(x) \\
    y=3x
  \end{array}
\right.$
NB, abbiamo trasformato $f(x)=0$ in $x=g(x)$
\\ \\
Dopo aver trovato le soluzioni, ne determino degli intervalli (ragionevolmente piccoli). Supponiamo che la soluzione $\alpha$ sia contenuta all'interno dell'intervallo $[0;0,5]$ (dal grafico sappiamo dov'è ma non sappiamo il relativo valore), iniziamo ad eseguire i seguenti step:
\begin{enumerate}
\item Trovo il punto medio delle due tabelle $x_i$;
\item Calcolo l'immagine del relativo punto $f(x_i)$
\item Calcolo la \textit{\textbf{toll}}, tolleranza facendo $\mid b_i-a_i\mid$
\item Assegno il valore $x_i$ ad \textit{a} o \textit{b} in base a questo modo:
\begin{itemize}
\item se $f(a_i) * f(x_i) < 0$ allora $b_{i+1} = x_i$ mentre \textit{a} resta invariato  
\item se  $f(a_i) * f(x_i) > 0$ allora $a_{i+1} = x_i$ mentre \textit{b} resta invariato  
\end{itemize}
\end{enumerate}


\newpage
\begin{table}[!h]
\centering
\begin{tabular}{|c|c|c|c|c|c|}
$i$ & $a_i$ & $b_i$ & $x_i$ & $f(x_i)$ & $\mid b_i-a_i \mid$ \\ 
\hline
0 & 0 (-) & 0.5 (+) & 0.25  & -0,218912421 & 0.5000 \\
& & & & &\\
1 & 0.25 (-) & 0.5 (+) & 0.375  &  0.194492378 & 0.2500 \\
& & & & &\\
2 & 0.25 (-) & 0.375 (+) & 0.3125 & -0.01467948 & 0,1250 \\
& & & & & \\
3 & 0.3125 (-) & 0.0375 (+) & 0.34375 & 0.089752536 & 0.0625
\end{tabular}
\caption{Tabella esecuzione Bisezione}
\end{table}


\noindent
Per semplificare la tabella, vicino al punto $a_i$ e $b_i$ è stato inserito un + o - tra parentesi. Questo indica se la rispettiva immagine ( $f(a_i)$ e $f(b_i)$) è positiva o negativa. Questo aiuta a semplificare il controllo con $f(x_i)$

\newpage

\subsection{Metodo di Newton}
\label{Metodo di Newton}
 La formula per calcolare l'iterata successiva è la seguente:
$$x_{n+1}=x_n-\frac{f(x_n)}{f'(x_n)}$$

\textbf{Esempio time}: Sia $f(x)=sin(x)+x-\pi$ e $x_0$=2 come valore di partenza, si determini una soluzione approssimata $x_3$.
\\
\\
Sappiamo che $f'(x)=cos(x)+1$

\begin{table}[h!]
\centering
\begin{tabular}{|c|c|c|c|c|}
$n$ & $x_n$ & $f(x_n)$ & $f'(x_n)$ &$-f(x_n)/f'(x_n)$ \\ 
\hline
0 & 2 & -0.232295226 & 0.583853163 & 0.397865834\\
& & & &\\
1 & 2.397865834 & -0.06691458 & 0.26409513 & 0.252571789 \\
& & & &\\
2 & 2.650437619 & -0.019510336 & 0.118211319 & 0.165046259 \\
& & & & \\
3 & 2.81548387 & / & / & /
\end{tabular}
\end{table}
\noindent
I test d'arresto per questo metodo sono:
\begin{itemize}
\item Una tolleranza \textbf{toll} tale che $\mid x_{n+1}-x_n \mid$ <toll
\item Numero massimo di iterazioni \textit{n}
\end{itemize}

\pagebreak

\subsection{Metodo della secante}
\label{Metodo della secante}
Metodo da utilizzare se $f'(x)$ non è disponibile oppure il metodo di Newton non può essere utilizzato. La formula per calcolare l'iterata successiva è la seguente:
$$x_{n+1}=x_n-f(x_n)*\frac{x_n-x_{(n-1)}}{f(x_n)-f(x_{n-1})}$$
\\ 
\textbf{Esempio Time}: prendiamo come funzione di partenza $f(x)=4cos(x/2)-3,5-x$ nell'intervallo $I=[0;1]$. I punti di partenza $x_0$ e $x_1$ corrispondono agli estremi degli intervalli.

\begin{center}
\begin{table}[h!]
\begin{tabular}{|c|c|c|c|c|c|}

$n$ & $x_n$ & $f(x_n)$ & $f(x_n)-f(x_{n-1})$ &$x_n-x_{n-1}$ & $x_{n+1}-x_n$ \\ 
\hline
0 & 0 & 0,5 & / & / & 1\\
& & & & &\\
1 & 1 & -0,98966975243 & -1,489669752 & 1 & -0.664355136 \\
& & & & &\\
2 & 0,335644863 & 0.108158481 & ... & ... & ...
\end{tabular}
\caption{Tabella esecuzione Seccante}
\end{table}
\end{center}

I test d'arresto sono come quelli di Newton:
\begin{itemize}
\item Tolleranza \textbf{toll}
\item Numero massimo di iterazioni
\end{itemize}

\newpage
\section{Matrici (Metodi diretti)}
\label{Matrici (Metodi diretti)}

Lo scopo di questo paragrafo è la risoluzione di sistemi attraverso l'uso delle matrici. Il sistema solitamente viene scomposto in un'equazione \textit{A}\textbf{x}=\textbf{b} dove:
\begin{itemize}
\item $A^{n*n}$=matrice contente le espressioni
\item \textbf{x}=vettore delle incognite
\item \textbf{b}=vettore dei termini noti
\end{itemize}

\noindent
Il concetto dei metodi diretti si basa sulla \textit{fattorizzazione}:
\begin{itemize}
\item Trasformo A nel prodotto di due matrici: $A = BC$
\item Il sistema passa da $Ax = b$ a $BC x = b$
\item Si pone $y = Cx$
\item Si risolvono i due sistemi $B y = b$ e $C x = y$ 
\end{itemize}



\subsection{Gauss}
\label{Gauss}
Permette di trasformare la matrice A in una matrice U triangolare superiore del tipo \textit{$A = LU$} con L triangolare inferiore composta dai soli moltiplicatori di A.
\\
\\
Per ottenere A diminuita attraverso Gauss si possono usare due modi:
\subsubsection{Gauss senza pivoting}
\label{Gauss senza pivoting}
Metodo "semplice e veloce" rispetto alla sua controparte. Ci si basa sul pivot che ci troviamo (pivot = primo elemento non nullo del gradino della matrice) e si azzerano i valori sotto il pivot. Ad esempio:
$$ A =
\begin{bmatrix}
\fbox{2} & 1 & 3 \\
4 & 3 & 10 \\
-2 & 1 & 7 
\end{bmatrix}
$$
\noindent
Il pivot in questo caso è 2 (posizione 1,1).
\\
\\
Ora determino i \textbf{moltiplicatori} i quali mi permettono di ottenere la colonna 0. In questo caso i moltiplicatori sono:
\begin{itemize}
\item $l_{21}$ = moltiplicatore della riga 2 con la riga 1 = $a_{21}/a_{11}$ = 4/2 = 2
\item $l_{31}$ = moltiplicatore della riga 3 con la riga 1 = $a_{31}/a_{11}$ = -2/2 = -1
\end{itemize}

Ora, per ogni riga sotto quella del pivot, eseguo la sottrazione della riga con quella del pivot moltiplicata per il moltiplicatore. In questo caso si esegue:
\begin{itemize}
\item \textbf{Per la riga 2 }: $\alpha_{2i}-(l_{21}*\alpha_{1i})$
\item \textbf{Per la riga 3 }: $\alpha_{3i}-(l_{31}*\alpha_{1i})$
\end{itemize}

Dopo aver fatto ciò dovremmo ottenere una matrice
$$ A^{(1)}
\begin{bmatrix}
2 & 1 & 3 \\
0 & \fbox{1} & 4 \\
0 & 2 & 10 
\end{bmatrix}
$$
\noindent
Non abbiamo ancora finito dato che A non è ancora triangolare superiore. 
\\
Si riesegue lo stesso procedimento di prima:
\begin{itemize}
\item Determino il pivot: 1 (posizione 2,2);
\item Calcolo i moltiplicatori, in questo caso è solo uno
	\begin{itemize}
		\item $l_{32}$ =  moltiplicatore della riga 3 con la riga 2 = $a_{32}/a_{22}$
	\end{itemize}
\item eseguo la sottrazione della riga 3 con la riga 2 moltiplicata per il moltiplicatore: $\alpha_{3i}-(l_{32}*\alpha_{2i})$
\end{itemize}

\noindent
Alla fine del procedimento otteniamo la matrice ridotta \textbf{U}
$$ U =
\begin{bmatrix}
2 & 1 & 3 \\
0 & 1 & 4 \\
0 & 0 & 2 
\end{bmatrix}
$$
\noindent
e la matrice dei moltiplicatori \textbf{L}
$$ L =
\begin{bmatrix}
1 & 0 & 0 \\
2 & 1 & 0 \\
-1 & 2 & 1 
\end{bmatrix}
$$

\noindent
Notate che che $l_{ij}$ corrisponde a dire "Moltiplicatore in riga \textbf{i} e colonna \textbf{j}".

\subsubsection{Gauss con pivoting}
\label{Gauss con pivoting}
Il metodo di Gauss \textbf{con} pivoting è leggermente diverso da quanto visto prima. È più sicuro del metodo precedente dato che evita situazioni con pivot nullo.\\\\
Cosa cambia da Gauss senza pivoting:
\begin{itemize}
\item È necessario effettuare un controllo prima di eseguire il calcolo del moltiplicatore che potrebbe portare ad uno scambio di righe;
\item Viene inserita una nuova matrice P definita "Matrice di permutazione". È una matrice che, a partire dalla matrice identità, contenente tutti gli scambi effettuati durante l'applicazione del metodo in modo da interpretare il pivoting come un prodotto di matrici. La sua definizione è la seguente: $P = P_n x P_{n-1} x ... x P_1$  (zero scambi ne consegue che P = I = Matrice d'identità).
\item La matrice finale non è più A = L x U ma \textbf{P x A = L x U}.
\end{itemize}

\noindent

Prendiamo come esempio la seguente matrice

$$ A =
\begin{bmatrix}
1 & 1 & 1 \\
2 & -3 & -1 \\
1 & 2 & -1 
\end{bmatrix}
$$

\noindent
Ora non si calcolano subito i moltiplicatori ma si determina il \textit{miglior pivot}. Per farlo si possono usare due tecniche di pivoting:
\begin{itemize}
\item \textbf{Pivoting parziale}: supponendo che \textbf{\textit{r$\geq$k }}
$$ |a_{rk}| = \max_{\substack{k\leq i\leq n}} |a_{ik}|$$
Quello che dice: il miglior pivot è l'elemento assoluto più alto della colonna K a partire dalla diagonale $a_{kk}$
\item \textbf{Pivoting totale}: solo a titolo informativo, consiste nell'effettuare la ricerca del miglio pivot considerando anche le altre colonne. Data la sua difficoltà nella gestione della matrice, viene preferito il pivoting parziale.
\end{itemize}

Tornando all'esempio, al primo step si osserva la colonna 1 e il miglior pivot è $a_{21}$ = 2.
Ora si effettua lo scambio della riga 1 con la riga 2 e si ottiene la seguente matrice

$$ A =
\begin{bmatrix}
2 & -3 & -1 \\
1 & 1 & 1 \\
1 & 2 & -1 
\end{bmatrix}
$$

Dato che abbiamo effettuato uno \textbf{scambio}, salviamolo in un "pezzo" della matrice di permutazione. Chiamiamolo $P_1$ che è uguale a 
$$ P_1 =
\begin{bmatrix}
0 & 1 & 0 \\
1 & 0 & 0 \\
0 & 0 & 1 
\end{bmatrix}
$$

Se moltiplichiamo A con $P_1$ possiamo infatti notare che abbiamo ottenuto la stessa matrice.

Da qua si calcolano i due moltiplicatori:
\begin{itemize}
\item \textbf{Per la riga 2 }:  $a_{21}/a_{11}$ = 1/2
\item \textbf{Per la riga 3 }:  $a_{31}/a_{11}$ = 1/2
\end{itemize}

Si effettuano i calcoli e si ottiene:
$$ A^{(1)} =
\begin{bmatrix}
2 & -3 & -1 \\
0 & 5/2 & 3/2 \\
0 & \fbox{7/2} & -3/2 
\end{bmatrix}
$$

Al secondo step si osserva la colonna 2 e il miglior pivot è $a_{32}$ = 7/2. Si effettua un altro scambio tra righe, ottenendo la seguente matrice:
$$ A^{(1)} =
\begin{bmatrix}
2 & -3 & -1 \\
0 & 7/2 & -3/2 \\
0 & 5/2 & 3/2
\end{bmatrix}
$$

e la seguente matrice di permutazione
$$ P_2 =
\begin{bmatrix}
1 & 0 & 0 \\
0 & 0 & 1 \\
0 & 1 & 0 
\end{bmatrix}
$$

\textbf{NB}: anche se nella matrice originale A la seconda riga corrispondeva alla prima, a noi non ce ne frega niente dato che viene considerata sempre la nuova permutazione di A.
\\ \\
Calcoliamo il moltiplicatore:
\begin{itemize}
\item \textbf{Per la riga 3 }:  $a_{32}/a_{22}$ = 5/7
\end{itemize}

Otteniamo così 
$$ U =
\begin{bmatrix}
2 & -3 & -1 \\
0 & 7/2 & -3/2 \\
0 & 0 & 3/2
\end{bmatrix}
$$
$$ L =
\begin{bmatrix}
1 & 0 & 0 \\
1/2 & 1 & 0 \\
1/2 & 5/7 & 1 
\end{bmatrix}
$$

Per quanto riguarda P invece, si ottiene dalla moltiplicazione delle varie matrici di permutazione, in questo caso $P_2$ e $P_1$. Quindi P = \\
\begin{center}
$
\begin{bmatrix}
1 & 0 & 0 \\
0 & 0 & 1 \\
0 & 1 & 0 
\end{bmatrix}
$
x
$
\begin{bmatrix}
0 & 1 & 0 \\
1 & 0 & 0 \\
0 & 0 & 1 
\end{bmatrix}
$
=
$
\begin{bmatrix}
0 & 1 & 0 \\
0 & 0 & 1 \\
1 & 0 & 0 
\end{bmatrix}
$.
\end{center}

\textbf{NB PARTE 2}: se viene applicato il pivoting parziale dopo aver già determinato dei moltiplicatori, \underline{bisogna cambiare la posizione dei moltiplicatori} \underline{delle righe coinvolte}.\\\\
\textbf{Esempio time}: abbiamo la seguente matrice:
$$ A =
\begin{bmatrix}
4 & 3 & 10 \\
2 & 1 & 3 \\
-2 & 1 & 7
\end{bmatrix}
$$
\noindent
Dopo il primo step abbiamo la matrici:\\

$$ A^{(1)} =
\begin{bmatrix}
4 & 3 & 10 \\
0 & -1/2 & -2 \\
0 & 5/2 & 12 
\end{bmatrix}
$$
\noindent

$$ L =
\begin{bmatrix}
1 & 0 & 0 \\
1/2 & 1 & 0 \\
- 1/2 & l_{32} & 1 
\end{bmatrix}
$$

Allo step 2 notiamo che c'è da eseguire il pivoting parziale. Avendo coinvolto le righe 2 e 3 è necessario (prima di calcolare il moltiplicatore) scambiare \textbf{TUTTI} i moltiplicatori della riga 2 con quelli della riga 3. Otteniamo così: \\

$$ A^{(2)} =
\begin{bmatrix}
4 & 3 & 10 \\
0 & 5/2 & 12 \\
0 & -1/2 & -2 
\end{bmatrix}
$$
\noindent
$$ L =
\begin{bmatrix}
1 & 0 & 0 \\
- 1/2 & 1 & 1 \\
1/2 & l_{32} & 0 
\end{bmatrix}
$$
\noindent
Il procedimento si è svolto anche nell'esempio precedente ma, essendo i due valori uguali, non si notava benissimo. \\
Ora, perché $l_{32}$ non viene coinvolto? Non viene coinvolto perché $l_{32}$ fa \textit{già parte della nuova permutazione}.

\subsection{Risoluzione dei sistemi}
\label{Risoluzione dei sistemi}
Dopo aver applicato le eliminazioni di Gauss (con o senza pivoting) non abbiamo ancora finito di risolvere i nostri sistemi lineari dato che abbiamo solo ottenuto una forma semplificata. Per risolvere i sistemi $Ax = b$ è necessario eseguire altri due step:
\begin{itemize}
\item \textbf{Se usato Gauss CON pivoting}: Dopo aver ottenuto i due sistemi triangolari U e L. Bisogna risolvere:
\begin{enumerate}
\item $L*y = P*b$
\item $U*x = y$
\end{enumerate}
\item \textbf{Se usato Gauss SENZA pivoting}: Dopo aver ottenuto i due sistemi triangolari U e L. Bisogna risolvere:
\begin{enumerate}
\item $L*y = b$ (NB P = I)
\item $U*x = y$
\end{enumerate}
\item \textbf{Se matrici aumentate multiple}: Partendo da una matrice aumentata multipla (A$\mid$b$_1$b$_2$...b$_n$), dopo aver applicato Gauss (con o senza pivoting) ed ottenuto (U$\mid$y$_1$y$_2$...y$_n$) basta risolvere gli n sistemi
\begin{itemize}
\item $U*x = y_1$
\item $U*x = y_2$
\item ...
\item $U*x = y_n$
\end{itemize}
\end{itemize}
\noindent

\subsection{Calcolo dell'inversa}
\label{Calcolo dell'inversa}
Sia A una matrice invertibile, per calcolare l'inversa A$^{-1}$ seguite i seguenti step
\begin{itemize}
\item Partendo da A, uso Gauss (con o senza pivoting) per risolvere la matrice aumentata (A$\mid$I)
\item Ottengo (U$\mid$Y) dove Y = L$^{-1}$ x P e P = I se ho fatto Gauss senza pivoting
\item Dalla matrice aumentata (U$\mid$L$^{-1}$ x P), risolvo gli n sistemi triangolari ottenuti, dove n = numero delle colonne di Y.
\end{itemize}
\noindent

\textbf{Esempio time}: dato (U$\mid$Y) =
$$
\begin{bmatrix}
  -4 & 7 & 0 & \aug & 1 & 0 & 0 \\
  0 & 5/2 & 7 & \aug &1/2 & 1 & 0 \\
  0 & 0 & -2/5 & \aug & 1/2 & -1/5 & 1 
\end{bmatrix}
$$

Devo risolvere questi 3 sistemi:
\begin{center}
$
\begin{bmatrix}
-4 & 7 & 0 \\
0 & 5/2 & 7 \\
0 & 0 & -2/5 
\end{bmatrix}
$
x
$
\begin{bmatrix}
a_{11} \\
a_{12} \\
a_{13} \\
\end{bmatrix}
$
=
$
\begin{bmatrix}
1 \\
1/2 \\
1/2 \\
\end{bmatrix}
$
\end{center}


\begin{center}
$
\begin{bmatrix}
-4 & 7 & 0 \\
0 & 5/2 & 7 \\
0 & 0 & -2/5 
\end{bmatrix}
$
x
$
\begin{bmatrix}
a_{21} \\
a_{22} \\
a_{23} \\
\end{bmatrix}
$
=
$
\begin{bmatrix}
0 \\
1 \\
-1/5 \\
\end{bmatrix}
$
\end{center}


\begin{center}
$
\begin{bmatrix}
-4 & 7 & 0 \\
0 & 5/2 & 7 \\
0 & 0 & -2/5 
\end{bmatrix}
$
x
$
\begin{bmatrix}
a_{31} \\
a_{32} \\
a_{33} \\
\end{bmatrix}
$
=
$
\begin{bmatrix}
0 \\
0 \\
1 \\
\end{bmatrix}
$
\end{center}

In questo modo ho calcolato i componenti della matrice $A^{-1}$

\subsection{Metodo di Cholesky}
\label{Metodo di Cholesky}
Il metodo di Cholesky offre una semplificazione della matrice più veloce rispetto a quella di Gauss ma è applicabile solo se A è :
\begin{itemize}
\item \textbf{simmetrica}: $A = A^T$
\item \textbf{definita positiva}: $x^T A x > 0$ 
\end{itemize}
\noindent
I valori si calcolano direttamente con le seguenti formule:
\begin{itemize}
\item $l_{11}$ = $\sqrt[2]{a_{11}}$ \quad \quad \quad \textit{Primo elemento di L}
\item $l_{i1}$ = $a_{i1}/ l_{11} $ \quad \quad \textit{Elementi della prima riga di L con $i = 1\dots n$}
\item $l_{jj}$ = \textit{Elementi diagonali della diagonale L}, si ottengono con la seguente formula (j = $2\dots n$) $$(a - \sum_{k=1}^{j-1} l^2_{jk})^{1/2}$$ 
\item $l_{ij}$ = \textit{Resto degli elementi}, si ottengono con la seguente formula (j = $2\dots n$ e i = $j+1\dots n$) $$(a_ij-\sum_{k=1}^{j-1}l_ik*l_jk)/l_{jj}$$
\end{itemize}
Applicato questo metodo si ottiene una semplificazione $A = L * L^T$.
\\
Per risolvere il sistema ottenuto bisogna calcolare:
\begin{itemize}
\item L y = b
\item L$^T$ x = y
\end{itemize}

\subsection{Determinante}
\label{Determinante}
Il determinante si calcola in modi diversi a seconda del metodo utilizzato:
\begin{itemize}
\item \textbf{Gauss}: $$det(A) = (-1)^s * \prod_{i=1}^{n} a^{(i-1)}_{ii}$$ dove s indica il numero di permutazioni eseguite. In sostanza \\$det(A) = 1^{permutazioni} *$ \textit{moltiplicazione degli elementi diagonali di U}
\item \textbf{Cholesky}: $$det(A) = \prod_{i=1}^{n} l^2_{ii}$$ In sostanza $det(A) =$ \textit{moltiplicazione degli elementi diagonali di L al quadrato}
\end{itemize}


\newpage
\section{Matrici (Metodi di rilassamento classici)}
\label{Matrici (Metodi di rilassamento classici)}

Il concetto dei metodi di rilassamento si basa sulla \textit{decomposizione} di A nella seguente forma: $$A = M - N$$
\noindent Il sistema può essere scritto come $$Mx = Nx + b$$ e partendo da un $x_0$ vettore arbitrario, è possibile costruire le relative iterazioni
$$Mx_{k+1} = Nx_k + b$$
I metodi di rilassamento classici costruiscono M e N basandosi sulle seguenti 3 matrici:\\
\noindent una matrice diagonale costruita con gli elementi diagonali di A
 $$D = 
\begin{bmatrix}
a_{11} & 0 & 0 \\
0 & \dots &  0\\
0 & 0 & a_{nn} 
\end{bmatrix}
$$

una matrice triangolare inferiore costruita con gli elementi inferiori alla diagonale di A, tutto il resto zero

$$ -E =
\begin{bmatrix}
0 & & & &\\
a_{21} & 0 & & & \\
\dots  & \dots & \ddots & \\
\dots  &  \dots & \dots & \ddots \\
a_{n1} & \dots & \dots & a_{n,n-1} & 0 
\end{bmatrix}
$$

ed infine una matrice triangolare superiore costruita con gli elementi superiori alla diagonale di A, tutto il resto zero

$$ -F =
\begin{bmatrix}
0 & a_{12} & \dots & \dots & a_{1n} \\
 & 0 & \ddots &  & \dots \\
 &  & \ddots & \ddots  & \dots\\
 &  &  & \ddots & a_{n-1,n}\\
 &  &  &  & 0
\end{bmatrix}
$$

Otteniamo in questo modo $A = D-E-F$
\\ \\
Inoltre, ogni metodo ha una sua particolare matrice d'iterazione.

\subsection{Metodo Jacobi}
\label{Metodo Jacobi}
Con questo metodo, le matrici M e N sono definite nel seguente modo:
\begin{itemize}
\item $M = D$
\item $N = E + F$ Prestate attenzione che i valori delle matrici E ed F di default sono \textbf{Negativi} (-E e -F).
\end{itemize}

Dopo aver determinato M e N si può calcolare la matrice di iterazione $B_j$. Questa può essere calcolata in due modi:
\subsubsection{Calcolo diretto}
$$B_J =
\begin{bmatrix}
0 & -\frac{a_{12}}{a_{11}} & \dots & \dots & -\frac{a_{1n}}{a_{11}} \\
-\frac{a_{21}}{a_{22}} & 0 &  \ddots & & \vdots \\
\vdots & \ddots & \ddots & \ddots & \vdots \\
-\frac{a_{n1}}{a_{nn}} & \dots & \dots & -\frac{a_{n,n-1}}{a_{nn}} & 0
\end{bmatrix}
$$

\subsubsection{Calcolo con D,E,F}
\label{Calcolo con D,E,F}
$$ \boldsymbol{B_J} = M^{-1}*N = \boldsymbol{D^{-1}*(E+F)} $$
dove  $$D^{-1} = 
\begin{bmatrix}
\frac{1}{a_{11}} & 0 & 0 \\
0 & \dots &  0\\
0 & 0 & \frac{1}{a_{nn}}
\end{bmatrix}
$$

\subsubsection{Calcolo dell'iterata successiva}
\label{Calcolo dell'iterata successiva}
Per calcolare $x_{k+1}$ a partire da un $x_k$ arbitrario, basta eseguire il seguente conto $$x_{k+1} = B_J * x_k + q$$
dove $q = D^{-1}*b$

\subsection{Metodo Gauss-Seidel}
\label{Metodo Gauss-Seidel}
Con questo metodo, le matrici M e N sono definite nel seguente modo:
\begin{itemize}
\item $M = D - E$
\item $N = F$
\end{itemize}

In questo caso per calcolare la matrice d'iterazione c'è solo un modo: 
$$B_G = M^{-1}*N = (D-E)^{-1}*F $$
\noindent
\textbf{PRO TIP}: per calcolare l'inversa di M in questo caso si può usare il metodo descritto sul paragrafo \ref{Calcolo dell'inversa} ma essendo lungo ed a tratti macchinoso può portare ad errori. Fortunatamente, essendo una triangolare, è possibile sfruttare questo trick:
\\
\\
\noindent
Una proprietà della matrice identità è che, data una matrice A, $A^{-1}A = I$.
Prendendo in questo caso $(D-E) =
\begin{bmatrix}
1/2 & 1 & 0 \\
-2 & 3 & 0 \\
1 & -1 & 1 
\end{bmatrix}
$ 
\\ \\ \\
\noindent
Abbiamo che $(D-E)^{-1}*(D-E) = I$
\begin{center}
$
\begin{bmatrix}
2 & 0 & 0 \\
a & 1/3 & 0 \\
b & c & 1 
\end{bmatrix}
$
x
$
\begin{bmatrix}
1/2 & 0 & 0 \\
-2 & 3 & 0 \\
1 & -1 & 1 
\end{bmatrix}
$
=
$
\begin{bmatrix}
1 & 0 & 0 \\
0 & 1 & 0 \\
0 & 0 & 1
\end{bmatrix}
$
\end{center}
\noindent
Moltiplico le due matrici ed ottengo: \\
\begin{center}
$
\begin{bmatrix}
1 & 0 & 0 \\
a/2 - 2/3 & 1 & 0 \\
b/2 - 2c+1 & 3c+1 & 1 
\end{bmatrix}
$
=
$
\begin{bmatrix}
1 & 0 & 0 \\
0 & 1 & 0 \\
0 & 0 & 1
\end{bmatrix}
$
\end{center}
\noindent
Come si può notare, ora basta solo far corrispondere gli spazi con le incognite ai relativi valori (in questo caso 0). Per risolvere questo basta risolvere il seguente sistema:
$$\left\{
  \begin{array}{lr}
    a/2 -2/3 = 0 \\
    b/2 -2c+1 = 0 \\
    3c+1 = 0 
  \end{array}
\right.
$$

Ossia rendere $a_{21} = I_{21} $ , $a_{31} = I_{31}  $, $a_{32} = I_{32} $, alla fine sostituiamo i valori di $a,b$ e $c$ nelle rispettive posizioni ed è fatta.
\\ \\
\noindent
\subsubsection{Calcolo dell'iterata successiva gauss-seidel}
\label{Calcolo dell'iterata successiva gauss-seidel}
Per calcolare $x_{k+1}$ a partire da un $x_k$ arbitrario, basta eseguire il seguente conto $$x_{k+1} = B_G * x_k + q$$
dove $q = (D-E)^{-1}*b$

\newpage
\section{Interpolazione}
\label{Interpolazione}
Rientra nelle approssimazione di funzioni (ridurre $f(x)$ in una $g(x)$ più semplice). L'interpolazione consiste nell'approssimare una funzione e che passi per gli stessi punti della funzione originale.
\\
\\
In questo capitolo viene trattata \textbf{l'interpolazione polinomiale}: data una funzione $f(x)$ nella quale:
\begin{itemize}
\item siano noti $n+1$ punti (o nodi) $x_i$ distinti;
\item sia $f(x_i)$ ($i=0\dots n)$ i valori assunti dalla funzione nei punti dati
\item esista un indice $0\leq k\leq n$ tale che $f(k)\neq 0$ 
\end{itemize}
Allora esiste un polinomio $P_n(x_i) = f(x_i)$

\subsection{Forma di Lagrange}
\label{Forma di Lagrange}
Non ha vincoli nell'utilizzo. Per trovare il polinomio interpolatore $P_n(x)$ è necessario eseguire i seguenti step:
\begin{itemize}
\item Determinazione polinomio caratteristico di Lagrange  $$L^(n)_i(x) = \prod_{\substack{j=0 \\ j\neq i}}^{n} \frac{x-x_i}{x_i-x_j}$$
\item Determinazione polinomio interpolatore $$P_n(x)=\sum_{i=0}^{n}f(x_i)*L_i^n(x)$$
\end{itemize}
\noindent
\newpage
Per fare un esempio: siano noti 
\begin{table}[!h]
\centering
\begin{tabular}{|c|c c c c }
$x_i$ & -2 & 0 & 1 & 2 \\
\hline
$f(x_i)$ & 4 & 10 & 10 & 16
\end{tabular}
\caption{Tabella nodi iniziali}
\end{table}


\noindent quindi n = 3.
\\\\
Il polinomio d'interpolazione si trova quindi in questo modo: $$P_3(x) = f(x_0)*L^{(3)}_0  + f(x_1)*L^{(3)}_1 + f(x_2)*L^{(3)}_2 + f(x_3)*L^{(3)}_3$$

Ovviamente noi di base non sappiamo i valori reali dei vari $L^{(3)}$ dobbiamo prima calcolarceli:
\begin{itemize}
\item $L^{(3)}_0 = \frac{x-0}{-2+0} * \frac{x-1}{-2-1} * \frac{x-2}{-2-2}$ 
\item $L^{(3)}_1 = \frac{x+2}{0+2} * \frac{x-1}{0-1} * \frac{x-2}{0-2}$
\item $L^{(3)}_2 = \frac{x+2}{1+2} * \frac{x+0}{1+0} * \frac{x-2}{1-2}$
\item $L^{(3)}_3 = \frac{x+2}{2+2} * \frac{x+0}{2+0} * \frac{x-1}{2-1}$
\end{itemize}
\noindent
Una volta risolti i conti basta eseguire il conto precedente e si ottiene il seguente polinomio $$P_3(x) = x^3-x+10$$

\subsection{Forma di Nevile-Aitken}
\label{Forma di Nevile-Aitken}
Per utilizzare Nevile-Aitken non sono necessarie particolari condizioni.\\
La forma del polinomio interpolatore è la seguente: $$P_n(x) = T_n^{(0)}(x)$$
Per determinare il $T_n^{(0)}$ sono necessarie queste due formule:
\begin{itemize}
\item $T_0^{(i)} = f(x_i)$
\item $T_{k+1}^{(i)} = T_k^{(i+1)}(x) - \frac{x_{i+k+1}-x}{x_{i+k+1}-x_i} * (T_k^{(i+1)}(x) - T_k^{(i)}(x))$
\end{itemize}
\newpage
\textbf{Esempio time}: siano noti i seguenti nodi
\begin{table}[!h]
\centering
\begin{tabular}{|c|c c c }
$x_i$ & 0.1 & 0.2 & 0.4 \\
\hline
$f(x_i)$ & 2 & 1 & 0.5
\end{tabular}
\caption{Tabella nodi iniziali}
\end{table}


Per facilitare la comprensione: dato $T_k^{(i)}$, k indica la colonna ed i indica il nodo interpolato
La tabella di risoluzione è con il metodo di Nevile-Aitken:

\begin{table}[!h]
\centering
\begin{tabular}{|c|c|c|c|}
$x_i$ & $T_0^{(i)}$ & $T_1^{(i)}$ & $T_2^{(i)}$ \\ 
\hline
0.1 & 2 & $3-10x$ & $25x^2-17.5+3.5$ \\
& & & \\
0.2 & 1 & $1.5-2.5x$& / \\
& & & \\
0.4 & 0.5 & / & /
\end{tabular}
\caption{Tabella esecuzione Nevile-Aitken}
\end{table}

Alla fine di tutto $\boldsymbol{P_2(x) = 25x^2-17.5+3.5}$
\\ \\ 
Il procedimento si svolge colonna per colonna, calcolando prima la colonna k = 0, poi la colonna k = 1 ed infine la colonna k = 2 e dall'alto verso in basso. Si nota, ad esempio, che le parti di $T_1^(0)$ Necessita di $T_0^{(1)}$

\subsection{Formule di Newton}
\label{Formule di Newton}
Con le interpolazioni, Newton ha più formule a disposizione. Quelle che interessano a noi sono principalmente due: \textbf{Newton alle differenze divise} e \textbf{Newton alle differenze finite in avanti}.
\\
\\
\noindent Tutte queste formule portano (in un modo o nell'altro e con scritture diverse) al seguente polinomio interpolatore: $$P_n(x) = c_0 + \sum_{i=0}^{n} c_i \prod_{j=0}^{i-1}(x-x_j)$$

Le cose che cambiano da una formula all'altra è il modo in cui vengono scritti i coefficienti $c_i$

\subsubsection{Formula di Newton alle differenze divise}
\label{Formula di Newton alle differenze divise}
In questo caso $c_0$ dipende solo da $f(x_0)$, il coefficiente $c_1 da f(x_0) e f(x_1)$ e cosi via fino a $c_n$ che dipende da $f(x_0),f(x_1)\dots f(x_n)$.
\\ \\
In sostanza, si dice che $c_n = f[x_0,x_1,\dots ,x_n]$
e prende il nome di \textit{differenza divisa di f}. \\ \\
\noindent Queste differenze si basano su ordini dove \textit{Ordine n} e vengono indicate in questo modo:
\begin{itemize}
\item \textbf{Ordine 0}: $f[x_i] = f(x_i)$
\item \textbf{Ordine 1}: $f[x_i,x_j] = \dfrac{f[x_i] - f[x_j]}{x_i-x_j}$
\item \textbf{Ordine 2} $f[x_i,x_j,x_k] = \dfrac{f[x_i,x_j] - f[x_j,x_k]}{x_i-x_k}$
\item \textbf{Ordine n}: $f[x_0,\dots ,x_n] = \dfrac{f[x_0,\dots ,x_{n-1}] - f[x_1,x_n]}{x_0-x_n}$
\end{itemize}

Alla fine di tutto, il nostro polinomio interpolatore è  $$P_n(x) = f(x_0) + \sum_{i=0}^{n} f[x_0,\dots ,x_i] \prod_{j=0}^{i-1}(x-x_j)$$

\textbf{Esempio time}: siano noti i seguenti nodi
\begin{table}[!h]
\centering
\begin{tabular}{|c|c c c c}
$x_i$ & 3 & -2 & -1 & 2 \\
\hline
$f(x_i)$ & 4 & -2 & -1 & 2 
\end{tabular}
\caption{Tabella nodi iniziali}
\end{table}

Per risolvere il seguente esempio con Newton alle differenze divise si può ricorrere alla solita tabella

\begin{table}[!h]
\centering
\begin{tabular}{|c|c|c|c|c|}
$x_i$ & $f[x_i] = f(x_i)$ & $f[x_i,x_{i+1}]$ & $f[x_i,x_{i+1},x_{i+2}]$ & $f[x_i,x_{i+1},x_{i+2},x_{i+3}]$\\ 
\hline
3 & 4 & 6/5 & 1/20 & 1/20 \\
& & & & \\
-2 & -2 & 1 & 0 & / \\
& & & &\\
-1 & -1 & 1 & / & / \\
& & & &\\
2 & 2 & / & / & /
\end{tabular}
\caption{Tabella esecuzione Newton differenze divise}
\end{table}
\newpage
Alla fine di tutto scrivo la formula del polinomio: $$P_3 = 4 + \dfrac{6}{5}*(x-3)+\dfrac{1}{20}*(x-3)(x-2)+\dfrac{1}{20}(x-3)(x+2)(x+1)$$ il quale risulta $$P_3(x) = \dfrac{1}{20}x^3 + \dfrac{1}{20}x^2+\dfrac{4}{5}x-\dfrac{1}{5}$$

Nel caso venga aggiunto un nuovo nodo, il nuovo polinomio diventerebbe: $$P_{n+1} = P_n(x) + f[x_0,\dots ,x_{n+1}]\prod_{j=0}{n}(x-x_j)$$ ossia il vecchio polinomio più la nuova differenza divisa. Ovviamente bisogna rifare alcuni conti sulla tabella di esecuzione. In particolare, se eseguita, si noterà che i conti che non riguardano il nuovo nodo aggiunto \textbf{saranno uguali}.

\begin{table}[!h]
\centering
\begin{tabular}{|c|c|c|c|c|c|}
$x_i$ & $f[x_i] = f(x_i)$ & $f[x_i,x_{i+1}]$ & $f[x_i,x_{i+1},x_{i+2}]$ & $f[x_i,x_{i+1},x_{i+2},x_{i+3}]$ &$f[x_i,\dots ,x_{i+4}]$\\ 
\hline
3 & 4 & 6/5 & 1/20 & 1/20 & \textbf{4/15} \\
& & & & &\\
-2 & -2 & 1 & 0 & \textbf{-3/4} & /\\
& & & & &\\
-1 & -1 & 1 & \textbf{-3/2} & / & / \\
& & & & &\\
2 & 2 & \textbf{-1/2 }& / & / & / \\
& & & & & \\
0 & \textbf{3} & / & / & / & / 
\end{tabular}
\caption{Tabella esecuzione con nuovo nodo, valori in grassetto = valori diversi}
\end{table}

\newpage

\subsubsection{Formula di Newton alle differenze divise in avanti}
\label{Formula di Newton alle differenze divise in avanti}
Al contrario delle differenze divise, ci sono due vincoli che devono essere rispettati per usare questa formula:
\begin{itemize}
\item I nodi siano \textbf{ordinati} in senso crescente
\item I nodi siano \textbf{equidistanti}
\end{itemize}
Se sono equidistanti, allora abbiamo $x_i = x_0 +i*h$
\\ \\
\textbf{NB}: nel caso vi venga chiesto se per i nodi \{0, -1, -2\} è possibile applicare Newton con le differenze in avanti, la risposta è \textbf{può essere applicato se vengono riordinati nel seguente modo \{-2,-1, 0\} perché sono equidistanti ma non ordinati}
\\ \\
\noindent
Il nostro polinomio interpolatore è  $$P_n(x) = f(x_0) + \sum_{i=0}^{n} \dfrac{1}{i!*h^i}\Delta ^if(x_0)\prod_{j=0}^{i-1}(x-x_j)$$
\noindent
Formula per determinare i delta $$\Delta ^kf(x) = \Delta ^{k-1}f(x+h) - \Delta ^{k-1}f(x)$$
dove $\Delta ^0f(x_i)=f(x_i)$
\newpage
\textbf{Esempio Time}: siano dati i seguenti nodi con le rispettive immagini
\begin{table}[!h]
\centering
\begin{tabular}{|c|c c c}
$x_i$ & 0 & 1 & 2 \\
\hline
$f(x_i)$ & 0 & 1 & 3 
\end{tabular}
\caption{Tabella nodi iniziali equidistanti e ordinati (h = 1)}
\end{table}

La tabella delle differenze in avanti è la seguente

\begin{table}[!h]
\centering
\begin{tabular}{|c|c|c|c|}
$x_i$ & $\Delta ^0f(x_i) = f(x_i)$ & $\Delta ^1f(x_i) =\Delta ^0f(x_i+h) - \Delta ^0f(x_i) $ & $\Delta ^2f(x_i) =\Delta ^1f(x_i+h) - \Delta ^1(x_i)$ \\ 
\hline
0 & 0 & $\Delta ^1f(0) =\Delta ^0f(1) - \Delta ^0f(0)$ = 1 & $\Delta ^2f(0) =\Delta ^1f(0+1) - \Delta ^1(0)$ 2 \\
& & &\\
1 & 1 & $\Delta ^1f(x_i) =\Delta ^0f(2) - \Delta ^0f(1)$ = 2 & / \\
& & & \\
2 & 3 & / & / \\
\end{tabular}
\caption{Tabella esecuzione Newton alle differenze in avanti}
\end{table}

Alla fine di tutto $$P_2(x) = 0 + \dfrac{1}{1!1^1}*1*(x-0)+\dfrac{1}{2!1^2}*1(x-0)(x-1)$$
e $$\boldsymbol{P_2(x) = \dfrac{x^2+x}{2}}$$

Nel caso si volesse aggiungere un nuovo nodo, la soluzione è simile a quella precedente se non per i valori dei \textit{coefficienti} $$P_{n+1}(x) = P_n+\dfrac{1}{(n+1)!h^{n+1}}\Delta^{n+1}f(x_0)\prod_{j=0}^{n}(x-x_j)$$
E sempre come prima, si ricalcolano i valori con il nuovo nodo aggiunto. \\
\textbf{NB} Controllare sempre che il valore che si voglia aggiungere sia \textit{equidistante} ed inserito in modo \textit{ordinato} rispetto ai nodi presenti.

\subsection{Approssimazione ai minimi quadrati}
\label{Approssimazione ai minimi quadrati}
Usata principalmente quando i dati sono sperimentali e l'adozione di uno dei metodi precedenti potrebbe dare un risultati che non hanno un elevato grado di esattezza.
\\
\\
Dati i punti ed il loro peso $w$, ci si può ricondurre alla funzione $f(x)$ attraverso l'uso delle matrici determinando il valore dei coefficienti $a$ delle variabili $x$. In particolare, si tratta di risolvere il seguente sistema $$A^TA\boldsymbol{a}=A^Tb$$
\\ \\
\noindent \textbf{Esempio time} dati i seguenti punti con i loro relativi pesi:
\textbf{Esempio Time}: siano dati i seguenti nodi con le rispettive immagini
\begin{table}[!h]
\centering
\begin{tabular}{|c|c c c c c}
$x_i$ & 0 & 1/2 & 1 & 3/2 & 2 \\
\hline
$f(x_i)$ & 0 & 1/2 & 1 & 0 & -1  \\
\hline
$\omega$ & 1 & 16 & 1 & 16 & 1
\end{tabular}
\caption{Tabella nodi iniziali con i pesi}
\end{table}

I quali posso vederli sotto forma di vettore riga:

$$x = 
\begin{bmatrix}
0 & 1/2 & 1 & 3/2 & 2
\end{bmatrix}
$$

$$y = 
\begin{bmatrix}
0 & 1/2 & 1 & 0 & -1
\end{bmatrix}
$$

$$\omega = 
\begin{bmatrix}
1 & 16 & 1 & 16 & 1
\end{bmatrix}
$$

Gli step per calcolare il tutto sono i seguenti:
\begin{itemize}
\item Definisco la matrice $A$ nel seguente modo
$$ A =
\begin{bmatrix}
\sqrt[2]{w_0} & \sqrt[2]{w_0}*x_0 & \sqrt[2]{w_0}*x_0 &\\
\vdots & \vdots & \vdots \\
\vdots  & \vdots & \vdots & \\
\sqrt[2]{w_m} & \sqrt[2]{w_m}*x_m & \sqrt[2]{w_m}*x_m & \\
\end{bmatrix}
$$
\item Definisco la matrice $b$ nel seguente modo:
$$ b =
\begin{bmatrix}
\sqrt[2]{w_0}*f(x_0) \\
\vdots \\
\vdots  \\
\sqrt[2]{w_m}*f(x_m) \\
\end{bmatrix}
$$
\item Eseguo le moltiplicazioni $A^TA$ e $A^Tb$
\item (Eventualmente) Semplifico il sistema trovato dalle due moltiplicazioni con Gauss o simili
\item Determino i valori dei coefficienti $a$
\end{itemize}
\noindent
Tornando all'esempio: \\ 
\begin{center}

$ A = \begin{bmatrix}
1 & 0 & 0 \\
4 & 2 & 1 \\
1 & 1 & 1 \\
4 & 6 & 9 \\
1 & 2 & 1 \\
\end{bmatrix}
$
\hbox{}
$ b = \begin{bmatrix}
0 \\ 2 \\ 1 \\ 0 \\ -1
\end{bmatrix}
$
\end{center}
\noindent Calcoliamo ora i due sistemi: \\
\begin{center}
$ A^TA = \begin{bmatrix}
1 & 4 & 1 & 4 & 1 \\
0 & 2 & 1 & 6 & 2 \\
0 & 1 & 1 & 9 & 4   
\end{bmatrix} $
x
$\begin{bmatrix}
1 & 0 & 0 \\
4 & 2 & 1 \\
1 & 1 & 1 \\
4 & 6 & 9 \\
1 & 2 & 1 \\
\end{bmatrix}
$
=
$\begin{bmatrix}
35 & 35 & 45 \\
35 & 45 & 65 \\
45 & 65 & 99 
\end{bmatrix}
$
\end{center}

\begin{center}
$ A^Tb = \begin{bmatrix}
1 & 4 & 1 & 4 & 1 \\
0 & 2 & 1 & 6 & 2 \\
0 & 1 & 1 & 9 & 4   
\end{bmatrix} $
x
$ \begin{bmatrix}
0 \\ 2 \\ 1 \\ 0 \\ -1
\end{bmatrix}
$
=
$ \begin{bmatrix}
8 \\ 3 \\ -1
\end{bmatrix}$
\end{center}

\newpage
Ora posso calcolare i nuovi valori dei coefficienti $a$ risolvendo

\begin{center}
$\begin{bmatrix}
35 & 35 & 45 \\
35 & 45 & 65 \\
45 & 65 & 99 
\end{bmatrix}
$
x
$\begin{bmatrix}
a_1 \\ a_2 \\ a_3
\end{bmatrix}
$
=
$ \begin{bmatrix}
8 \\ 3 \\ -1
\end{bmatrix}
$
\end{center}

Risolvo usando la matrice aumentata (A$\mid$b) e, dopo aver usato Gauss, ottengo
$$\begin{bmatrix}
35 & 35 & 45 & \aug & 8\\
0 & 10 & 20 &\aug & -5\\
0 & 0 & 8/7 &\aug & -9/7
\end{bmatrix}
$$

A questo punto si risolve il sistema
$$\left\{
  \begin{array}{lr}
    35a_1 + 35a_2 + 45a_3 = 8 \\
    10a_2 + 20a_3 = -5 \\
    8a_3/7 = -9/7 
  \end{array}
\right.
$$

Il polinomio finale è $\boldsymbol{g(x) = a_1 + a_2x * a_3x^2}$ sostituendo ai rispettivi coefficienti i valori trovati nel sistema


\newpage
\section{Integrazione numerica}
\label{Integrazione numerica}
Metodi per l'approssimazione di un'integrale definito. Usati quando l'integrale non è determinabile per via analitica o quanto l'espressione è troppo complessa e soggetta a notevoli errori.
\\
\\
\noindent
Le formule che saranno utilizzate vengono definite \textit{composte} per il fatto che l'intervallo [a,b] viene suddiviso in m intervalli (non necessariamente uguali).
\subsection{Formula di trapezi composta}
\label{Formula di trapezi composta}
Dato un integrale $$\int_{a}^{b} f(x) dx$$ 
e posti gli intervalli uguali, calcoliamo $h = (b-a)/m$
\\
La formula di trapezi composta è la seguente $$\dfrac{h}{2} *[f(a) + 2 \sum_{i=1}^{m-1}f(a+ih) + f(b)]$$

\subsection{Formula di Cavalieri-Simpson}
\label{Formula di Cavalieri-Simpson}
\textbf{Condizione nell'uso = m dev'essere pari}\\
Dato un integrale $$\int_{a}^{b} f(x) dx$$ 
e posti gli intervalli uguali, calcoliamo $h = (b-a)/m$
La formula di Cavalieri-Simpson è la seguente: $$\dfrac{h}{3} *[f(a)+ 4 \sum_{i=1(dispari)}^{m-1}f(a+ih) + 2 \sum_{i=1(pari)}^{m-2}f(a+ih) + f(b)]$$

\subsection{Formula semplificata}
\label{Formula semplificata}
Da utilizzare dopo aver trovato una prima approssimazione \textbf{con trapezi composta}. Veloce perché ti permette di riutilizzare quanto precedentemente calcolato, riutilizzandolo e senza dover ricalcolare tutto da capo\\ \\
Dato un nuovo m (solitamente corrisponde al doppio di quello precedente), abbiamo che la nuova approssimazione è $$ T_{i+i} = \dfrac{h_{i+1}}{2} * S_{i+1}$$

Dove $$S_{i+i} = S_i + 2 \sum_{i=1(dispari)}^{m-1}f(a+ih) $$ ossia la valutazione dei nuovi intervalli sommata a quella precedentemente ottenuta \\ \\
mentre $$S_i =  2 \sum_{i=1}^{m-1}f(a+ih)$$ non è altro che la valutazione degli intervalli precedenti calcolati con trapezi composta
\end{document}